\documentclass[11pt]{amsart}

\usepackage{geometry}             % See geometry.pdf to learn the layout options. There are lots.
\geometry{letterpaper}            % ... or a4paper or a5paper or ... 
%\geometry{landscape}             % Activate for for rotated page geometry
%\usepackage[parfill]{parskip}    % Activate to begin paragraphs with an empty line rather than an indent

\usepackage{amssymb}
%\usepackage{graphicx}
%\usepackage{epstopdf}
%\DeclareGraphicsRule{.tif}{png}{.png}{`convert #1 `dirname #1`/`basename #1 .tif`.png}

\usepackage[]{hyperref}

\title{Project Outline}
\author{Juan Durazo, Brendan Horan, Arthur Mitrano}
\date{\today}  % Activate to display a given date or no date

\begin{document}
\maketitle

\section{Introduction}
We study how noise propagates in the \emph{Golub-Kahan iterative bidiagonalization (GKIB)} in the
problem 
\begin{equation} \label{eq:problem}
	Ax\approx b, \indent  b=b^{exact} + b^{noise} \in \mathbb{R}^n.
\end{equation}

In the above equation the noise level of $b^{noise}$ is unknown, but through this bidiagonalization
process, it is possible to determine it. The first part of this study will be to replicate results
from the numerical experiments done in \cite{bidiagonalization}. More specifically, we will find
at what iteration step the noise is revealed, and how we can find the noise level from this, as
done in \cite{bidiagonalization}.

The chosen test problem will be {\bf shaw}, so in \eqref{eq:problem}, $A$ is the blurring matrix and
$b^{exact}$ is the blurred image that is returned from the \texttt{shaw} function. We will generate 
$b^{noise}$ and add it to $b^{exact}$ to create $b$. Using this information, we apply the Golub-Kahan
iterative bidiagonalization using MATLAB code that we found on-line \url{http://www.cs.cas.cz/krylov}
to perform our numerical experiments. From this code we will obtain a sequence of real numbers 
$\alpha_j,\beta_j$ and vectors $s_j,w_j$ for $k=1,2,3,\ldots$ that are going to be use to determine
when the noise is revealed and how to get the noise level.


\section{Replicate results from main paper}
The first order of business will be to replicate the results from section 3 of our primary article. Here we will plot the left bidiagonalization vectors given by the Golub-Kahan iterative algorithm for the purpose of determining an appropriate number of basis vectors to retain in the solution. The smoothing and orthogonalization in the algorithm can also be interpreted as a step-by-step elimination of the dominant low frequency components, revealing the high frequency noise. We hope to explore the method presented in section 4 for approximating this noise level given the particular iteration, k_noise, and compare to the authors' results.

\section{Use NCP approach to obtain results}
We will attempt to find $k_{noise}$ using the \emph{Normalized Cumulative Periodogram (NCP)} approach.
This involves computing the cumulative periodogram of the basis vectors $s_k$ and seeing if as $k$ 
increases the noise becomes more visible in the basis vector $s_k$, that is, the cumulative periodogram
of those vectors will begin to approach the diagonal that connects the points with 0 and 1 ordinates.
 
In order to do this, we will need a way to measure how noise-like the cumulative periodogram is and at
what point we consider it to be noise. Two methods that we will consider for this measurement are the
deviations of the basis vectors from a straight diagonal "white noise" line and the other one will be
to asses the portion of the basis vectors that lies outside the \emph{Kolmogorov-Smirnov test}. 
The latter approach requires us to provide a confidence level for the white noise line and that is
something that we will experiment with. From this approach, we will investigate if we can find the
bidiagonalization iteration that is noise revealing. For more information on this topic we will check
\cite{peridograms}.


\section{Compare the two methods}
The two methods described above to find the moment where the noise is revealed will be compared. We will look
for differences and advantages of those methods, in special, how the predicted noise level changes with the 
$k_{noise}$ acquired by the Normalized Cumulative Periodogram and the method described in \cite{bidiagonalization}.
To get the noise level $\delta_{noise}$ we are going to use 
\begin{equation*}
	\delta_{noise} = \frac{\|b^{noise}\|}{\|b^{exact}\|} \approx \frac{1}{2}\rho_{k_{noise}}, \quad \text{where}
	\quad \rho_{k} = \left(\prod_{j=1}^k\frac{\alpha_j}{\beta_{j+1}}\right)^{-1},
\end{equation*}
as described on section 4.2 of \cite{bidiagonalization}.
Moreover, plots of the basis vectors $s_k$ will be provided to be clear at which iteration the noise is
revealed.

\section{More information}
\begin{sloppypar}
To find more information and source code that is current in development you can access the Github repository
\url{https://github.com/arthurmitrano/noisePropagationBidiagonalization}. You can also find 
versions of the report file there. No unauthorized code will be uploaded, since the project is open currently
public.
\end{sloppypar}

\begin{thebibliography}{1}
	\bibitem{bidiagonalization}  Hnetynkova, I. and Plesinger, M..
		\emph{The regularizing effect of the Golub-Kahan iterative bidiagonalization and revealing the
             noise level in the data.}
		BIT Numer Math (2009) 49: 669–696.
	\bibitem{periodograms} Rust, B. W. and O'Leary, D. P.. 
		\emph{Residual periodograms for choosing regularization parameters for ill-posed problems.}
		Inverse Problems 24 (2008) 034005 (30pp).
\end{thebibliography}

\end{document}  