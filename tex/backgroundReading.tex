\documentclass{article}

\usepackage{amsmath}
\usepackage{amsfonts}

\begin{document}
\title{Noise propagation on Golub-Kahan bidiagonalization process}

\maketitle

\section{Group details}
We have 3 group members:
\begin{itemize}
	\item Juan Durazo;
	\item Bredan Horan;
	\item Arthur Mitrano.
\end{itemize}

\section{Background reading}
After a brief discussion with group members and the class instructor, the following papers will be study:

\begin{enumerate}
	\item Hnetynkova, I. and Plesinger, M..
		\emph{The regularizing effect of the Golub-Kahan iterative bidiagonalization and revealing the noise level in the data.}
		BIT Numer Math (2009) 49: 669–696.
	\item Rust, B. W. and O'Leary, D. P.. 
		\emph{Residual periodograms for choosing regularization parameters for ill-posed problems.}
		Inverse Problems 24 (2008) 034005 (30pp).
	\item Rust, B. W. 
		\emph{Parameter Selection for Constrained Solutions to Ill-Posed Problems.}
		Computing Science and Statistics (2000), 32, 333-347.
	\item  Hansen, P. C. and Jensen, T. K..
		\emph{Noise propagation in regularizing iterations for image deblurring.}
		Electronic Transactions on Numerical Analysis. Volume 31, pp. 204-220, 2008.
	\item Jensen, T. K. and Hansen, P. C..
		\emph{Iterative regularization with minimum-residual methods.}
		BIT Numerical Mathematics (2007) 47: 103–120.
	\item Hansen, P. C., Kilmer, M. E. and Kjeldsen, R. H..
		\emph{Exploiting residual information in the parameter choice for discrete ill-posed problems.}
		BIT Numerical Mathematics (2006) 46: 41–59.
\end{enumerate}
%It might be necessary to read other papers or books for better comprehension of the subjects of the above articles.

\end{document}