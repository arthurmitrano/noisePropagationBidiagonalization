\documentclass[11pt]{amsart}
\usepackage{geometry}                % See geometry.pdf to learn the layout options. There are lots.
\geometry{letterpaper}                   % ... or a4paper or a5paper or ... 
\usepackage{graphicx}
\usepackage{amssymb}
\usepackage{epstopdf}
\DeclareGraphicsRule{.tif}{png}{.png}{`convert #1 `dirname #1`/`basename #1 .tif`.png}

\title{Final Report}
\author{Juan Durazo \\ Arthur Mitrano \\ Brendan Horan}
%\date{}                                           % Activate to display a given date or no date

\begin{document}
\maketitle
\section{ introduction}
\subsection{Problem} \indent \\
The problem we are seeking to solve is 
$$Ax=b, \indent A  \in  \mathbb{R}, \indent b = b^{exact} + b^{noise}$$ \\
In the above equation, $b^{noise}$ is a white noise vector with unknown level $\delta_{noise}$.
We investigate how noise enters the problem through $b$ and how it propagates into the 
core problem. Using this information, we can develop a stopping criteria for hybrid methods 
that are based on the Golub-Kahan bidiagonalization(GKb) process. Additionally, we 
can estimate the original noise level that was introduced into the problem with $b$.

\section{Finding $\bf k_{noise}$ using Normalized Cumulative Periododogram}
The Normalized Cumulative Periodogram(NCP) can be used to determine how "white-noise like" 
each $s_k$ is.


\end{document}  